%-----------------------------------------
% Template Author
% Michael Wright
%
% URL: http://github.com/mdwrigh2/resume
% This document is covered under the Creative Commons Attribution 3.0 Unported
% License
%-----------------------------------------

%!TEX TS-program = xelatex
%!TEX encoding = UTF-8 Unicode

\documentclass[11pt,letterpaper,sans,final]{moderncv}
\usepackage{fontspec}
\moderncvtheme[blue]{classic}

% Set percent of page to use and top and bottom margins
\usepackage[scale=0.8, top=.6in, bottom=.4in]{geometry}
% For ensuring 1 in margin:
%\usepackage[left=1in, right=.9in, top=.5in, bottom=.4in]{geometry}

\setlength{\hintscolumnwidth}{2cm} % Shrinks width of left column
\AtBeginDocument{\recomputelengths}

% Character encoding
\usepackage[utf8]{inputenc}

% Converts LaTeX specials (``quotes'' --- dashes etc.) to unicode
\defaultfontfeatures{Mapping=tex-text}

% Sets symbol for list items
\renewcommand{\listitemsymbol}{-}

% Suppress automatic page numbering
\nopagenumbers{}

%------------------------------------------

% Personal Information
\name{Michael James}{Murphy}
%\address{2537 Clubhouse Drive}{Wexford, PA 15090}
%\address{Carnegie Mellon University SMC 3373}{Pittsburgh, PA 15289}
\phone[mobile]{(724) 713-9797}
\email{mjmurphy@cmu.edu}
%\homepage{}

%\title{Michael J Murphy's Resume}

%------------------------------------------

\begin{document}

\makecvtitle

\vspace*{-1\baselineskip}

%\cvitem{\textbf{Objective:}}{A summer internship utilizing my programming
%skills, robotics experience, strong academic background, and broad interests
%in compilers, programming languages, robotics, and education.}

\section{Education}
  \href{http://coursecatalog.web.cmu.edu/schoolofcomputerscience/#curriculum-b.s.incomputerscience}
  {
    \cventry{May 2017}{Carnegie Mellon University}{Pittsburgh, PA}{QPA: 3.6}{}
      {Bachelors of Science in Computer Science, Minor in Music Technology}
    \cvline{Selected Coursework:}{
            Compiler Design,
            Machine Learning,
            Foundations of Programming Languages,
            Introduction to Computer Music,
            Probability and Computing,
            %Introduction to Computer Systems,
            Great Theoretical Ideas in Computer Science,
            %Parallel and Sequential Data Structures and Algorithms,
            %Principles of Functional Programming,
            %Principles of Imperative Computation
            % --- Music courses ---
            Production Audio,
            Sound Recording,
            Solfege,
            Harmony,
            Eurythmics
            }
  }
%  \cventry{June 2013}{North Allegheny Senior High School}{Wexford, PA}{QPA: 4.3}
%    {}{High School Diploma}

\section{Skills}
  \cvline{Languages:}{Python, OCaml, SML, Java, C, C\#, bash, JavaScript, HTML, \LaTeX}
  \cvline{Libraries:}{Jane Street Core, ROS, Django, Bootstrap, Qt}

\section{Internships}
  \cventry{Summer 2015}{Software Engineering Intern}{WhatsApp}
    {Mountain View, CA}{}{
      \begin{itemize}
         \item Worked on the Windows Phone team implementing networking
           protocols and improving reliability
%        \item Implemented the HTTP chunking protocol to route traffic through
%          a web proxy
      \end{itemize}
    }
\section{Teaching Experience}
  \cventry{Fall 2014 - Spring 2015}{Teaching Assistant}
    {15-122 Principles of Imperative Computation}{}{}
    {
      \begin{itemize}
        \item Independently taught weekly recitations on fundamental data
          structures and algorithms
      \end{itemize}
    }
  \cventry{Fall 2014, Fall 2015}{Teaching Assistant}{15-131 Great Practical
    Ideas in Computer Science}{}{}
    {
      \begin{itemize}
        \item Periodically lectured about 90 freshmen on bash, text editors,
          and other tools for computer scientists
      \end{itemize}
    }
  \cventry{Summer 2014}{Teaching Assistant}{15-110 Principles of Computing}{}{}
    {
      \begin{itemize}
        \item Prepared and gave a 5-10 minute daily lecture on
          introductory computer science topics
        \item Taught daily lab section of about 30 students, and held daily
          office hours
      \end{itemize}
    }
  \cventry{Summer 2016}{Teaching Assistant}{15-122 Principles of Imperative
    Computation}{}{}
    {
      \begin{itemize}
        \item As head TA, coordinated class and other TAs
      \end{itemize}
    }
  \cventry{}{Teaching Assistant}{15-150 Principles of Functional
    Programming}{}{}{}
  \cventry{Spring 2014 - Present}{Peer Tutor}{CMU Academic Development}
    {Pittsburgh, PA}{}{
    \begin{itemize}
      \item Assisted with homework and explained material from various courses
        on a weekly basis
    \end{itemize}
    }

\section{Research Projects (CORAL Research Group, Professor Manuela Veloso)}
  \cventry{Summer 2013}{Robot UI}{}{}{}{
    \begin{itemize}
      \item Designed and coded a new UI for the CoBot mobile service robots
        using ROS
      \item Implemented in a linux touch interface using Python and Qt
      \item Gained experience working with large foreign code bases and using
        third party libraries
    \end{itemize}}
  \href{http://data.cobotrobots.com}{
    \cventry{Summer 2014}{Robot Log Explorer Site}{data.cobotrobots.com}{}{}
    {
      \begin{itemize}
        \item Implemented a script to parse log data from CoBot mobile service
          robots into a database
        \item Designed a website using this database to search logs across many
          categories including coordinates visited, and to display and graph the
          results
        \item Gained experience learning new languages and tools and creating
          custom algorithms
      \end{itemize}}
  }
%  \cventry{Fall 2014 - Present}{Quadcopter Mobile Service Robot}{}{}{}{
%    \begin{itemize}
%      \item Working to adapt the existing grounded CoBot mobile service robots
%        to a new quadcopter platform
%      \item Researching the issues which arise when migrating robots to a new
%        physical platform
%    \end{itemize}}

\section{Projects}
  \href{http://ait.michaeljamesmurphy.com}{
    \cventry{Fall 2014}{Adaptive Interval Tutor}{ait.michaeljamesmurphy.com}{}{}
    {
      \begin{itemize}
        \item Worked with three people to create a web app to train users' ear
          for musical intervals
        \item Designed a probabilistic intelligent tutoring system to add
          adaptive difficulty
      \end{itemize}
    }
  }

%\section{Honors}
%  \cvline{Spring 2014}{School of Computer Science Dean's List}
\end{document}

